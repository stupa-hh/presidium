%!TEX program = lualatex
\documentclass[ngerman,headheight=70pt]{scrartcl}
\usepackage[ngerman]{babel}
\usepackage{blindtext}
\usepackage[utf8]{luainputenc}
\usepackage{scrlayer-scrpage}
\usepackage[T1]{fontenc}
\usepackage{fontspec}
\usepackage{multicol}
\usepackage{enumitem}
\usepackage[bottom=2cm,footskip=8mm]{geometry}
\setmainfont{Source Sans Pro}

\setlength{\parindent}{0pt}
\setlength{\parskip}{6pt}

\pagestyle{plain.scrheadings}
\thispagestyle{scrheadings}
\lohead{\vspace*{-1.5cm}\includegraphics[scale=0.3]{images/uhh-logo.pdf}}
\rohead{\\Präsidium des\\Studierendenparlaments}

\lofoot{Universität Hamburg · Präsidium des Studierendenparlaments\\\hline Von-Melle-Park 5 · D-20146 Hamburg}
\cofoot{}

\renewcommand{\thesubsubsection}{\arabic{subsubsection}.}

\begin{document}
    UHH · StuPa-Präsidium · Von-Melle-Park 5 · D-20146 Hamburg

    %%% PAGE ONE %%%%
    \section*{Protokoll der 1. Sitzung des Studierendenparlaments vom 14. April 2016}

    \textbf{Protokoll: Geoffrey}\\
    \textbf{Ort: HWP-Hörsaal}\\
    \textbf{Beginn: 18:25 Uhr}\\
    \textbf{Ende: ~0:50 Uhr}

    \vspace{0.5cm}
    \begin{tabular}{ll}
        CampusGrün (14 Sitze): & Laura Franzen, Geoffrey Youett, Elena Rysikova, \\
                                & Philipp Droll, Yasemin Günther, Melf Johannsen,\\
                                & Tahnee Herzig, Mario Moldenhauer, Jim Martens,\\
                                & Svenja Horn, Mirzo Ulugbek Khatamov, Armin Günther,\\
                                & Martin Sievert \\
                                &\\
        Entschuldigt: & Freya Schmitz (CampusGrün)\\
                                &\\
        Unentschuldigt abwesend: &\\
                                &\\
        Rücktritte: & Lasse Kleinlützum (MIN-Liste) \(\rightarrow\) Thea Wahlers \\
                    & Kerstin Riecke (CampusGrün) \(\rightarrow\) Blerta Vila \(\rightarrow\) Martin Sievert\\
                    & Elvis Milojevic (HWP-Liste) \(\rightarrow\) Samet Gunay
    \end{tabular}

    %%% Vorgeschlagene Tagesordnung %%%
    \underline{Vorgeschlagene Tagesordnung}
    \begin{enumerate}[label={\textbf{Top \theenumi}},leftmargin=*]
        \item Geschäftsordnung (V1617-005, V1617-009)
        \item Wahl des StuPa-Präsidiums
        \item RIS-Wahl
            \begin{enumerate}
                \item Bestätigung der Wahlniederschrift (V1617-003)
                \item Bestätigung der Referentinnen (V1617-003A01)
            \end{enumerate}
        \item Queer-Wahl
            \begin{enumerate}
                \item Bestätigung der Wahlniederschrift (V1617-006)
                \item Bestätigung der Referent*innen
            \end{enumerate}
        \item Wahl des Satzungs-, Wahlordnungs-, und Geschäftsordnungsausschusses
        \item Wahl des Ausschusses gegen Rechts
        \item Wahl des Haushaltsausschusses
        \item Wahl des Wirtschaftsrats
        \item
            \begin{enumerate}
                \item Verfahren zur Wahl zum Ältestenrat
                \item Wahl des Ältestenrats
            \end{enumerate}
        \item
            \begin{enumerate}
                \item Rechenschaftsbericht des amtierenden AStA (60 Min.)
                \item Fragen und Diskussion
                \item Entlastung des AStA
            \end{enumerate}
        \item Wahl des neuen AStA-Vorstandes
            \begin{enumerate}
                \item Diskussion VS-Thesen
                \item Wahl des AStA-Vorstandes
            \end{enumerate}
        \item Bestätigung der AStA-Referent*innen
        \item #uhh hilft (V1617-007)
        \item Gegen Rechts (V1617-008)
        \item Verschiedenes
    \end{enumerate}

    \subsection*{TOP 0 Formalia}

    \subsubsection{Geschäftsbericht Präsidium}

    \subsubsection{Anfragen an das Präsidium}

    \subsubsection{Geschäftsbericht AStA}

    \subsubsection{Anfragen an den AStA}

    \subsubsection{Dringlichkeitsanträge des AStA}

    \subsubsection{Aktuelle Stunde (falls entsprechender Antrag vorliegt)}

    \subsubsection{Feststellung der endgültigen Fassung des Teils B der Tagesordnung}

    \subsubsection{Feststellung der Beschlussfähigkeit}

    \subsubsection{Genehmigung der Protokolle der vorangegangenen Sitzungen}

    \subsection*{TOP 1 Gegen Rechts}

    \subsection*{TOP 2 Geschäftsordnung}

    \subsection*{TOP 3 Wahl des StuPa-Präsidiums}

    \subsection*{TOP 4 RIS-Wahl}

    \subsection*{TOP 5 Queer-Wahl}

    \subsection*{TOP 6 Wahl des Satzungs-, Wahlordnungs- und Geschäftsordnungsausschusses}

    \subsection*{TOP 7 Wahl des Ausschusses gegen Rechts}

    \subsection*{TOP 8 Wahl des Haushaltsausschusses}

    \subsection*{TOP 9 Wahl des Wirtschaftsrats}

    \subsection*{TOP 10 Ältestenrat}
\end{document}
